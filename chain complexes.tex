\documentclass[12pt]{article}

\usepackage{tikz}
\usetikzlibrary{arrows,chains,matrix,positioning,scopes}

\makeatletter
\tikzset{join/.code=\tikzset{after node path={%
\ifx\tikzchainprevious\pgfutil@empty\else(\tikzchainprevious)%
edge[every join]#1(\tikzchaincurrent)\fi}}}
\makeatother

\tikzset{>=stealth',every on chain/.append style={join},
         every join/.style={->}}
\tikzstyle{labeled}=[execute at begin node=$\scriptstyle,
   execute at end node=$]

\begin{document}

\begin{center}
\begin{tikzpicture}
	\matrix (m) [matrix of math nodes, row sep=3em, column 						sep=3em,nodes={anchor=center}]
    {	& 0 & 0 & 0 &  \\
    		\cdots & A_{n+1}  & A_n  & A_{n-1}  & \cdots \\
     	\cdots & B_{n+1} & B_n & B_{n-1} & \cdots \\ 
      	\cdots & C_{n+1} & C_n & C_{n-1} & \cdots \\
      	& 0 & 0 & 0 & \\		};
      
	% Vertical arrows 
	{ [start chain] \chainin (m-1-2);
	\chainin (m-2-2);
	\chainin (m-3-2) [join={node[right, labeled] {i}}];
	\chainin (m-4-2) [join={node[right, labeled] {j}}];
	\chainin (m-5-2);
	}
	
	{ [start chain] \chainin (m-1-3);
	\chainin (m-2-3);
	\chainin (m-3-3) [join={node[right, labeled] {i}}];
	\chainin (m-4-3) [join={node[right, labeled] {j}}];
	\chainin (m-5-3);
	}
	{ [start chain] \chainin (m-1-4);
	\chainin (m-2-4);
	\chainin (m-3-4) [join={node[right, labeled] {i}}];
	\chainin (m-4-4) [join={node[right, labeled] {j}}];
	\chainin (m-5-4);
	}

    
    % Horizontal arrows
  	{ [start chain] \chainin (m-2-1);
    \chainin (m-2-2);
    \chainin (m-2-3) [join={node[above,labeled] {\partial}}];
    \chainin (m-2-4) [join={node[above,labeled] {\partial}}];
    \chainin (m-2-5); }
    
    { [start chain] \chainin (m-3-1);
    \chainin (m-3-2);
    \chainin (m-3-3) [join={node[above,labeled] {\partial}}];
    \chainin (m-3-4) [join={node[above,labeled] {\partial}}];
    \chainin (m-3-5); }
    
    { [start chain] \chainin (m-4-1);
    \chainin (m-4-2);
    \chainin (m-4-3) [join={node[above,labeled] {\partial}}];
    \chainin (m-4-4) [join={node[above,labeled] {\partial}}];
    \chainin (m-4-5); }
\end{tikzpicture}
\end{center}

\end{document}