\documentclass[12pt]{article}

\usepackage{tikz}
\usetikzlibrary{decorations.markings}
\usetikzlibrary{calc}

\begin{document}

%		3-GON

\tikzset{->-/.style n args={2}{decoration={
  markings,
  mark=at position #1 with {\arrow{#2}}},postaction={decorate}}}

% The bullet modifier is creating the points at each node
\begin{tikzpicture}[bullet/.style={circle,inner sep=1.5pt,fill}]
	
	% Computes y-coordinate of A since non integral	
	\pgfmathsetmacro\result{sqrt(3)} 
	% Path itself does not draw anything- the foreach does that
	\path	(1,\result) node[bullet,label=above:$A$](A){}
  			(0,0) node[bullet,label=below:$B$](B) {}
  			(2,0) node[bullet,label=below:$C$](C) {};
  
	% The "\X/\Y/\Z" format means changing three variables
	% at once.
  	\foreach \X/\Arrow/\tiploc[remember=\X as \LastX (initially C)]	in {A/stealth/.5,B/stealth/.58,C/stealth/.55} {\draw[line width=1pt,->-={\tiploc}{\Arrow}](\LastX) -- (\X);}

\end{tikzpicture}


\end{document}